% ----------------------------------------

\subsection{General Problem}

% ----------------------------------------

\begin{frame}

\frametitle{Problem formulation}

The purpose of the \textbf{optimization algorithm} is to find the \textit{optimal budget} for each campaign in order to maximize the \textit{profit}, which is defined as the difference between the profits gained from selling products and the advertisement expenses.

Basically, it is a maximization problem subject to an obvious constraint: the sum of the daily budget can't be greater then the overall budget.

The optimization problem can be expressed as follows:
\begin{displaymath}
	F=\max_{\substack{x_i\in B}} \sum_{i=0}^n \alpha_i(x_i)p_i-x_i ~\text{  s.t.  } \sum_{i=0}^n x_i\leq B
\end{displaymath}

\end{frame}

% ----------------------------------------

\subsection{Algorithm}

% ----------------------------------------

\begin{frame}

\frametitle{Code}
\framesubtitle{Functioning}

\vspace*{-1em}

Given a \textbf{budget assignment problem} over the matrix $c$, which is assumed to be divided into $n$ rows (campaigns) and $m$ columns (budget fractions from 0 to $B$ \textit{maximum budget}).

We solve this problem via \textbf{DP} and we start by setting up \textbf{two extra tables} to store the \textit{best values} and their \textit{relative allocations}, then we iterate over each row and column finding the best index for each cell by taking into consideration the sum between the current and the previous row matching budgets; we store the best values and their indexes in the corresponding tables.

The optimal allocation lies at the end of the of the updated table, however, in order to obtain a final allocation which fully utilizes the whole budget we trace back the best values table \textbf{greedily} adding sub-optimal allocations until our budget runs out.

\end{frame}

% ----------------------------------------

\begin{frame}[fragile]

\frametitle{Code}
\framesubtitle{Example}

\begin{lstlisting}[style=Python, basicstyle=\tiny, numbers=none, framexrightmargin=-20pt]
dp = [c[0]]
allocs = [np.arange(m)]

for i in range(1, n):
	next_dp_row = []
	next_alloc_row = []
	for j in range(m):
		prev_row_r = dp[i - 1][j::-1]
		curr_row = c[i][: j + 1]

		row_sum = prev_row_r + curr_row

		# The best index is the index which we should select for
		# the maximum profit, it will be what we store in
		# allocation, and we will also store the value of the
		# row_sum at best_index to dp[i][j]
		best_index = np.argmax(row_sum)

		next_alloc_row.append(best_index)
		next_dp_row.append(row_sum[best_index])

	allocs.append(np.array(next_alloc_row))
	dp.append(np.array(next_dp_row))

dp = np.array(dp)
allocs = np.array(allocs)
# ...
\end{lstlisting}

\end{frame}

% ----------------------------------------

\begin{frame}[fragile]

\frametitle{Code}
\framesubtitle{Example}

\begin{lstlisting}[style=Python, basicstyle=\tiny, numbers=none, framexrightmargin=-20pt]
# ...
best_dp_index = np.argmax(dp[-1])

# As we stored the allocation of what the dp would mean, we now need to
# trace back through the allocs array to create the final allocation
last_alloc = allocs[-1][best_dp_index]

final_allocs = [last_alloc]
remaining_budget = m - last_alloc

for i in range(n - 2, -1, -1):
	# Based on the remaining_budget, we take the max of the remaining
	# possible cumulative values that we have in the dynamic
	# programming table
	sub_dp = dp[i, :remaining_budget]
	sub_best_index = np.argmax(sub_dp) if len(sub_dp) > 0 else 0

	next_alloc = allocs[i][sub_best_index]
	final_allocs.append(next_alloc)

	remaining_budget = remaining_budget - next_alloc

# As we constructed the the final allocation from the back to front,
# we need to reverse it before returning it
return np.array(list(reversed(final_allocs)))
\end{lstlisting}

\end{frame}

% ----------------------------------------
