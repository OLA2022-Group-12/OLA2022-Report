% ----------------------------------------

\subsection{Assignment}

% ----------------------------------------

\begin{frame}

\frametitle{Project Scope}
\framesubtitle{Assignment}

The focus of this project is to optimize the budget allocated for the \textbf{advertisement campaigns} of 5 different items in order to maximize the profit of an \textit{e-commerce website} that sells products to the public.

Each page has a \textbf{primary product} and some \textbf{secondary products}, each user that lands on a page has a possibility to buy the primary prodcut and/or proceed to one of the secondaries with a given probability, the process repeats.

\end{frame}


% ----------------------------------------

\begin{frame}

\frametitle{General hypoteses}

The main general hypoteses that were given to us are:
\begin{itemize}[label={-}]
    \item For every \textit{primary product}, the \textit{secondary products} to display and their order is fixed.
    \item The price of every product is fixed and it is equal to the margin.
    \item By clicking on a specific ad, the user lands on the corresponding \textit{primary product}.
    \item No bidding optimization needs to be performed.
\end{itemize}

\end{frame}

% ----------------------------------------

\subsection{Technical Approach}

% ----------------------------------------

\begin{frame}

\frametitle{Learner interface}
\framesubtitle{Description}

In order for our project to have a more general outline, we decided to model a \textbf{generic learner interface} to standardize how our agents should be expected to behave while learning the budget distribution for a set of subcampaigns in a specific scenario defined by the environment.

In particular, each learner is characterized by the \textit{learn} and \textit{predict} actions.
Every different type of learner will also receive a customized set of information filtered by the \textit{masked environment} (in line with each project step) and potentially employs different algorithms to learn and predict the various budgets.

\end{frame}

% ----------------------------------------

\begin{frame}[fragile]

\frametitle{Learner interface}
\framesubtitle{Code}

%Arguments: interactions: the interactions of the users which led to the given reward reward: the reward obtained from the environment based on the prediction given, needed for the tuning of internal properties done by the learner
%prediction: array containing the previous budget evaluation of the learner
%
%Arguments: data: up-to-date, complete or incomplete environment information that is used by the learner in order to make the inference
%Returns: a tuple containing a list of values (corresponding to the budgets inferred given the knowledge obtained by the learner until now) and a list of features (referring to which particular customers were the budgets aimed for, if None, the budgets apply to all the customers)

\begin{lstlisting}[style=Python, basicstyle=\tiny, numbers=none, framexrightmargin=-20pt]
class Learner(ABC):

  # Updates the learner's properties according to the reward received.
  \@@abstractmethod@@/
  def learn(self, interactions: List[Interaction], reward: float,
            prediction: np.ndarray):
    pass

  # Makes an inference about the values of the budgets for the subcampaigns
  # from the information got over time and the current environment
  \@@abstractmethod@@/
  def predict(self, data: MaskedEnvironmentData) ->
              Tuple[np.ndarray, Optional[List[List[Feature]]]]:
    pass

  # Creates a figure and plots showing the status of learning progress
  \@@abstractmethod@@/
  def show_progress(self, fig: plt.Figure):
    pass

\end{lstlisting}

\end{frame}

% ----------------------------------------

\begin{frame}

\frametitle{Comparing results}

For each different learner, when possible, we will show our results through the comparison of the rewards obtained by each algorithm against the rewards achieved (in the same enviroment setting) from the \textbf{Clairvoyant learner} and the Stupid learner.
This, aside from theoretical formulations, will help us to define \textbf{upper bounds} and \textbf{lower bounds} for different solutions.

The \textit{Stupid Learner} always subdivides the budget \textit{equally} between products.

In particular, the \textbf{Clairvoyant learner} makes its "prediction" with full knowledge of the problem by identifying the \textbf{optimal superarm} (which corresponds to the optimal assigment of budgets) and placing it in a deterministic environment to collect an estimate of the best rewards achievable.

\end{frame}

% ----------------------------------------

\begin{frame}

\frametitle{Comparing results}
\framesubtitle{Clairvoyant learner}

To estimate the \textbf{optimal superarm} we compute the \textbf{reward multipliers} (how much profit a single product brings on average) through hypotetical deterministic graph walks, performed by group of users following the expected values of the distributions defined on the graph edges.
In addition, to remove any \textbf{non-determinism}, $\alpha$-values are applied directly without any \textit{Dirichlet noise} applied to them.

Given this definition of \textbf{clairvoyant reward} it may happen that, during simulation runs, the \textit{learer reward} surpasses the \textit{clairvoyant reward}; this is an expected and negligible behavior caused by the absence of noise in the clairvoyant formulation.

\end{frame}

% ----------------------------------------

\begin{frame}

\frametitle{Comparing results}
\framesubtitle{Regret}

In order to evaluate the performance of our learers against the \textit{clairvoyant result} we use the \textbf{cumulated regret} formulation:

\begin{displaymath}
    R_T(L) := T\mu_{a^*} - \sum_{t = 1}^T \mu_{a_t}
\end{displaymath}

Usually we consider averaged results between multiple runs in order to reduce variance in the outcomes.

\end{frame}

% ----------------------------------------

\begin{frame}

\frametitle{Gaussian processes}

\textbf{Gaussian Processes} (\textbf{GPs}) are a powerful tool to tackle regression problems: by defining a \textbf{kernel} with its parameters and feeding the GP with the arms pulled by the bandit and their rewards it's possible to obtain a probability distribution over the outcome in form of arrays of \textbf{means} and \textbf{sigmas}.

Different bandit algorithms use those values in different ways as the \textbf{GPUCB} uses them to build a confidence bound and \textbf{GPTS} uses them directly to output estimations.

Each \textbf{GP} has been instantiated with the kernel: $Ck(\theta) * RBF(\ell)$ where $Ck$ is the $\theta$-constant kernel and $RBF$ is the \textit{Radial Basis Function} kernel of length $\ell$, defined as:
\begin{displaymath}
exp \left( - \frac{ \lVert x - x' \rVert^2 }{ 2\ell^2} \right)
\end{displaymath}

\end{frame}

% ----------------------------------------

\begin{frame}

\frametitle{Gaussian processes}
\framesubtitle{Regret}

\vspace*{-3em}

\begin{displaymath}
    R_T \le \sqrt{ \frac{2\Lambda^2}{log(1+\frac{1}{\sigma^2})} CBT \sum_{k=1}^C \gamma_{k, T} } \text{ with probability of } 1 - \delta
\end{displaymath}
where:
\begin{displaymath}
    B = 8log\left( 2\frac{T^2MC}{\delta} \right)
\end{displaymath}

In our case $C=5$ (number of subcampaigns), $\delta=?$ (regret bound confidence), $\gamma_{k, T}=? ~\forall k \text{ subcampaigns}$ (information gain at $T$), $\Lambda=?$ (Lipschitz constant of the problem), $M=?$ (number of arms) and $\sigma=?$ (maximum variance of the GP).


\end{frame}

% ----------------------------------------

\begin{frame}

\frametitle{UCB1 formulation}
\framesubtitle{Outline}

In the \textbf{UCB1} algorithm, every \textit{arm} is associated with an \textbf{upper confidence bound} which provides an \textit{estimation} of the reward gained by playing that specific arm.

When each arm has been played at least once in order to have a baseline for its reward, at every trial, the arm with the \textbf{highest upper confidence bound} is pulled (\textit{optimism in the face of uncertainty}).
After having collected the realization of the reward for the chosen arm, its \textit{upper confidence bound} is updated accordingly.

\end{frame}

% ----------------------------------------

\begin{frame}[fragile]

\frametitle{UCB1 formulation}
\framesubtitle{Formalities}

Code snippet that calculates the \textbf{upper bound}:

\begin{lstlisting}[style=Python, basicstyle=\tiny, numbers=none, framexrightmargin=-20pt]
def estimation(self):
  upper_bounds = (self.means + self.confidence * 1.96 * self.sigmas)
                   * self.normalize_factor
  return upper_bounds
\end{lstlisting}

\todo{GPUCB and remove Theoretical regret}

The \textit{theoretical regret} given by this kind of algorithm is bounded by:

\begin{displaymath}
    R_T(UCB1) \le \sum_{a:\mu_a < \mu_a^*} \frac{4log(T)}{\Delta_a} + 8\Delta_a
\end{displaymath}

where $\Delta_a$ is the difference in expected reward between the \\ optimal arm $a^*$ and the arm $a$: $\Delta_a = \mu_a^* - \mu_a$

\end{frame}

% ----------------------------------------

\begin{frame}

\frametitle{TS formulation}
\framesubtitle{Outline}

Some aspects of \textbf{TS} are similar to the \textbf{UCB1} since both are bandit algorithms, the main difference is that in \textbf{TS} every arm is associated to a $prior \beta distribution$.

Every arm has a prior on its \textbf{expected value} based on its mean distribution.
After drawing a sample according to the corresponding prior distribution, the algorithm chooses the arm with the \textbf{best sample}, then, it updates the distribution of the chosen arm according the observed realization.

\end{frame}

% ----------------------------------------

\begin{frame}[fragile]

\frametitle{TS formulation}
\framesubtitle{Formalities}

\todo{GPTS and remove Theoretical regret}

The \textit{theoretical regret} given by this kind of algorithm is bounded by:

\begin{displaymath}
    R_T(TS) \le (1+\epsilon) \sum_{a:\mu_a < \mu_a^*} \frac{\Delta_a(log(T)+log(log(T)))}{K*\Lambda(\mu_a*, \mu_a)} + C (\epsilon, \mu_{a_1} , ... , \mu_{a_{\left|A\right|}})
\end{displaymath}

where $K*\Lambda(mu_a*,mu_a)$ is the \textbf{Kullback-Leibler divergence} and, as before, $\Delta_a$ is defiened as the difference in expected reward between the optimal arm $a^*$ and the arm $a$: $\Delta_a = \mu_a^* - \mu_a$

\end{frame}

% ----------------------------------------
