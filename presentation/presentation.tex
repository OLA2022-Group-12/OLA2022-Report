%----------------------------------------------------------------------------------------
%	PACKETS AND CONFIGURATION
%----------------------------------------------------------------------------------------

\documentclass[11pt]{beamer}
\setbeamercovered{transparent}

\usepackage{title}  		% Title settings for the presentation
\usepackage{parskip}    	% Paragraph indent & skip
\usepackage{xcolor}     	% More colors
\usepackage{enumitem}		% Better lists
\usepackage{graphicx}  		% 'graphics' package interface
\usepackage{listings}   	% Code sections formatting}
%\usepackage{pdftexcmds}
%\usepackage{minted}
%\usepackage[utf8]{inputenc}
%\usepackage[T1]{fontenc}
%\usepackage{pythontex}
\usepackage{verbatim}
\usepackage{amsmath}

% Custom colors
\definecolor{AnnotationPurple}{RGB}{103, 13, 173}
\definecolor{CommentGreen}{RGB}{29, 204, 49}

% Python code formatting
% Use it by creating an environment like \begin{lstlisting}[style=Python] ... \end{lstlisting}
% Since annotations aren't tagged as keywords I have inserted a manual delimiter for them,
% it works as follows: \@@AnnotationText@@/
\lstdefinestyle{Python}{
    language = Python,
    basicstyle = \footnotesize\ttfamily,
    commentstyle = \textcolor{CommentGreen},
    stringstyle = \textcolor{orange},
    showstringspaces = false,
    keywordstyle = \textcolor{blue},
    moredelim = [is][\textcolor{AnnotationPurple}]{\\@}{@@\/},   % Annotations
    numberstyle = \scriptsize\ttfamily\textcolor{gray},
    numbers = left,
    frame = single,
    frameround = tttt,
    framexleftmargin = 10pt,
    framexrightmargin = 0pt
}

% HELPER PACKAGES (TODO: REMOVE IN FINAL) %
\usepackage{todonotes}
\presetkeys{todonotes}{inline}{}
\usepackage{blindtext}
% HELPER PACKAGES (TODO: REMOVE IN FINAL) %

\usetheme{Antibes} % THEME

%----------------------------------------------------------------------------------------
%	DOCUMENT
%----------------------------------------------------------------------------------------

\begin{document}

% TITLE
\frame{\titlepage}

% Table of Contents
\begin{frame}{Presentation Outline}
\begin{tiny}
    \tableofcontents[hideallsubsections]
\end{tiny}
\end{frame}

% Environment and Simulation ------------------------------------------------------------

\AtBeginSection[]
{
\begin{frame}{}
    \tableofcontents[sections={\thesection}]
\end{frame}
}

% ----------------------------------------

\section{Introduction}

%-----------------------------------------

\begin{frame}

\frametitle{Project Scope}
\framesubtitle{Assignment}

The focus of this project is to optimize the budget allocated for the \textbf{advertisement campaigns} of 5 different items in order to maximize the profit of an \textit{e-commerce website} that sells products to the public.

Each page has a \textbf{primary product} and some \textbf{secondary products}, each user that lands on a page has a possibility to buy the primary prodcut and/or proceed to one of the secondaries with a given probability, the process repeats.

\end{frame}


%-----------------------------------------

\begin{frame}

\frametitle{Project Scope}
\framesubtitle{Techniques}

The optimization has to be perfomed on different scenarios and using different algorithms.

We had to direct our focus on bandit algorithms combining Gaussian Processes with Upper Confidence Bound and Thompson Sampling algorithms.

For each different scenario the performance of the learners based on these algorithm is evaluated against the Clayrvoiant and the Stupid Learner.

\end{frame}

%-----------------------------------------

\begin{frame}

\frametitle{General hypoteses}

The main general hypoteses that were given to us are:
\begin{itemize}[label={-}]
    \item For every \textit{primary product}, the \textit{secondary products} to display and their order is fixed.
    \item The price of every product is fixed and it is equal to the margin.
    \item By clicking on a specific ad, the user lands on the corresponding \textit{primary product}.
\end{itemize}

\end{frame}

% ----------------------------------------

\section{Environment and Simulation}

%-----------------------------------------

\subsection{Overview}

%-----------------------------------------

\begin{frame}

\frametitle{Assumptions}
\framesubtitle{E-commerce website}

For this project, we are required to design an \textbf{Environment} that satisfies various constraints both on the e-commerce site's properties and on the users' behavior; in addition, since most of the tasks were generic, we had to come up with some of our own assumptions.

In particular we want to underline the following assumptions for the e-commerce website:
\begin{itemize}[label={-}]
    \item The website has unlimited stock for the 5 different items.
    \item Actions on the various webpages are \textbf{perfectly observable} by the e-commerce website.
\end{itemize}

\end{frame}

% ----------------------------------------

\begin{frame}

\frametitle{Assumptions}
\framesubtitle{Users}

...and we assume that the users present the following behaviors:
\begin{itemize}[label={-}]
    \item Every day, there is a random number (subject to noise) of potential new users.
    \item The \textit{reservation price} for each user is always over the single unit. \todo{?}
    \item The users can activate parallel paths while on the website.
    \item The number of items that a user will buy is a random variable, independent from any other variable.
\end{itemize}

\todo{insert text somewhere}
%The behavior of a user is modelled as a graph where nodes represent product pages and weights represent the probabilities for the user to click from the primary item of the page to one of the secondaries.

\end{frame}

%-----------------------------------------

\subsection{Environment structure}

%-----------------------------------------

\begin{frame}[fragile]

\frametitle{Environment}
\framesubtitle{Structure}

The environment is modelled as a python dataclass containing the following attributes:

\begin{lstlisting}[style=Python, basicstyle=\tiny, numbers=none, framexrightmargin=-20pt]
    # The total budget to subdivide
    total_budget: int

    # Probability of every class to show up. They must add up to 1
    class_ratios: List[float]

    # Features associated to every class
    class_features: List[List]

    # Price of the 5 products
    product_prices: List[float]

    # List of class parameters for each class and product,
    # implemented as list of lists of UserClassParameters.
    # Each class has distinct parameters for every product
    classes_parameters: List[List[UserClassParameters]]
\end{lstlisting}

\end{frame}

% ----------------------------------------

\begin{frame}[fragile]

\frametitle{Environment}
\framesubtitle{Structure}

\begin{lstlisting}[style=Python, basicstyle=\tiny, numbers=none, framexrightmargin=-20pt]
    # Lambda parameter, which is the probability of osserving the
    # next secondary product according to the project's assignment
    lam: float

    # Max number of items a customer can buy of a certain product.
    # The number of bought items is determined randomly with
    # max_items as upper bound
    max_items: int

    # Products graph's matrix. It's a empty matrix, should be
    # initialized with populate_graphs
    graph: np.ndarray

    # List that constains for every i+1 product the secondary i+1
    # products that will be shown in the first and second slot
    next_products: List[Tuple[int, int]]

    # Controls randomness of the environment
    random_noise: float
\end{lstlisting}

\end{frame}

% ----------------------------------------

\begin{frame}

\frametitle{Environment}
\framesubtitle{Masked Environment}

Alongside the \textbf{Environment} we define a \textbf{Masked Environment} with the purpose of \textit{hiding crucial information} to the learners since each type of learner should only have access to a subset of all the information available in the environment dictated by the type of learner.

The masked environment isn't strictly needed in the project since the learners could easily ignore the extra information, however, we wanted to face the problem with an approach aimed towards reusability and extendability and in this case (as in many others down the line) we opted for a more \textbf{generalizable} solution.

\todo{we talk about learners without metioning them before, we need to introduce them in the introductory section}

\end{frame}

% ----------------------------------------

\begin{frame}

\frametitle{Users}
\framesubtitle{User parameters}

We modelled the $\alpha$-functions, which compute the expected value of interactions on a product given a fixed budget, as \textbf{exponential functions}.

In particular their \textit{upper bound} represents the maximum expected number of interactions possible while the \textit{maximum useful budget} is the amount of budget after which any budget increase would not lead to a ratio increase.

\end{frame}

% ----------------------------------------

\begin{frame}

\frametitle{Users}
\framesubtitle{User classes}

\begin{itemize}[leftmargin=*, label={$\circ$}]
    \item Users are subdivided in classes based on their \textit{2 binary features} for a total of \textit{3 different classes}.
    \item In particular, each user class is defined by its $\alpha$-functions (one for each product plus the one for the non-strategic competitor) which define the probability of landing on a given product page.
    \item Each $\alpha$-function is defined by the values: \textbf{reservation price}, \textbf{upper bound} and \textbf{maximum useful budget}
\end{itemize}

\end{frame}

% ----------------------------------------

\subsection{Randomness in the Environment}

% ----------------------------------------

\begin{frame}

\frametitle{Randomness in the Environment}
\framesubtitle{Non determinism}

\begin{itemize}[leftmargin=*, label={$\circ$}]
    \item For the sake of representing a real scenario, most of the values that are not known a priori are \textit{randomly generated} and every variable that evolves through time without our direct control has elements of randomness to it (for instance, each day we randomly get the number of active total users in our scenario by using a gaussian distribution with tunable mean and standard deviation).
    \item Even though most of the randomness is tunable and controlled through seeded generators, there are still \textbf{impactful elements of non determinism} (i.e. the Dirichlet distribution) that are not possible to control in any way.
\end{itemize}

\end{frame}

% ----------------------------------------

\begin{frame}

\frametitle{Randomness in the Environment}
\framesubtitle{Consequences}

\todo{how does non determinism affect clairvoyant results?}

\end{frame}

% ----------------------------------------

\subsection{Simulation}

% ----------------------------------------

\begin{frame}

\frametitle{Daily Simulation}
\framesubtitle{Purpose}

\begin{itemize}[leftmargin=*, label={$\circ$}]
    \item The \textbf{Simulation} class is the main engine that brings together \textit{learners} and \textit{environment} by making them interact with each other while offering an interface to customize the execution.
    \item The basic idea of the simulation is to simulate a real scenario day by day using the environment to generate interactions with the website according to the budgets that the current learner proposed and then, feed the results back to the learner to make it actually learn.
    \item Repeating the simulation execution step for each day until an arbitrary \textbf{time horizon} is reached grants us all the data needed to evaluate the performance of our learner.
\end{itemize}

\end{frame}

% ----------------------------------------

\begin{frame}

\frametitle{Daily Simulation}
\framesubtitle{Example}

\todo{insert generic learner execution plot over n days}

\end{frame}

% Optimization Algorithm ----------------------------------------------------------------

\AtBeginSection[]
{
\begin{frame}{}
    \tableofcontents[sections={\thesection}]
\end{frame}
}

% ----------------------------------------

\section{Optimization Algorithm}

% ----------------------------------------

\subsection{General Problem}

% ----------------------------------------

\begin{frame}

\frametitle{Problem formulation}

The purpose of the \textbf{optimization algorithm} is to find the \textit{optimal budget} for each campaign in order to maximize the \textit{profit}, which is defined as the difference between the profits gained from selling products and the advertisement expenses.
Basically, it is a maximization problem subject to an obvious constraint: the sum of the daily budget can't be greater then the overall budget.

The optimization problem can be expressed as follows:
\begin{displaymath}
F=\max_{\substack{x_i\in B}} \sum_{i=0}^n \alpha_i(x_i)p_i-x_i \ s.t. \sum_{i=0}^n x_i\leq B
\end{displaymath}

\end{frame}

% ----------------------------------------

\subsection{Algorithm}

% ----------------------------------------

\begin{frame}

\frametitle{Code}
\framesubtitle{Functioning}

\todo{talking about the code in words}

\end{frame}

% ----------------------------------------

\begin{frame}

\frametitle{Code}
\framesubtitle{Example}

\todo{snippets of code}

\end{frame}

% Uncertain alpha functions -------------------------------------------------------------

\AtBeginSection[]
{
\begin{frame}{}
    \tableofcontents[sections={\thesection}]
\end{frame}
}

% ----------------------------------------

\section{Uncertain $\alpha$-functions}

% ----------------------------------------

\subsection{General Approach}

% ----------------------------------------

\todo{This entire subsection will be relocated in the introduction after merging}

\begin{frame}

\frametitle{Learner interface}
\framesubtitle{Description}

In order for our project to have a more general outline, we decided to model a \textbf{generic learner interface} to standardize how our agents should be expected to behave while learning the budget distribution for a set of subcampaigns in a specific scenario defined by the environment.

In particular, each learner is characterized by the \textit{learn} and \textit{predict} actions.
Every different type of learner will also receive a customized set of information filtered by the \textit{masked environment} (in line with each project step) and potentially employs different algorithms to learn and predict the various budgets.

\end{frame}

% ----------------------------------------

\begin{frame}[fragile]

\frametitle{Learner interface}
\framesubtitle{Code}

%Arguments: interactions: the interactions of the users which led to the given reward reward: the reward obtained from the environment based on the prediction given, needed for the tuning of internal properties done by the learner
%prediction: array containing the previous budget evaluation of the learner
%
%Arguments: data: up-to-date, complete or incomplete environment information that is used by the learner in order to make the inference
%Returns: a tuple containing a list of values (corresponding to the budgets inferred given the knowledge obtained by the learner until now) and a list of features (referring to which particular customers were the budgets aimed for, if None, the budgets apply to all the customers)

\begin{lstlisting}[style=Python, basicstyle=\tiny, numbers=none, framexrightmargin=-20pt]
class Learner(ABC):

  # Updates the learner's properties according to the reward received.
  \@@abstractmethod@@/
  def learn(self, interactions: List[Interaction], reward: float,
            prediction: np.ndarray):
    pass

  # Makes an inference about the values of the budgets for the subcampaigns
  # from the information got over time and the current environment
  \@@abstractmethod@@/
  def predict(self, data: MaskedEnvironmentData) ->
              Tuple[np.ndarray, Optional[List[List[Feature]]]]:
    pass

  # Creates a figure and plots showing the status of learning progress
  \@@abstractmethod@@/
  def show_progress(self, fig: plt.Figure):
    pass

\end{lstlisting}

\end{frame}

% ----------------------------------------

\begin{frame}

\frametitle{Comparing results}

For each different learner, when possible, we will show our results through the comparison of the rewards obtained by each algorithm against the rewards achieved (in the same enviroment setting) from the \textbf{Clairvoyant learner} and the Stupid learner.
This, aside from theoretical formulations, will help us to define \textbf{upper bounds} and \textbf{lower bounds} for different solutions.

In particular, the \textbf{Clairvoyant learner} makes its "prediction" with full knowledge of the problem while the stupid learner always subdivides \textit{equally} the budget between products.

\end{frame}

% ----------------------------------------

\begin{frame}

\frametitle{Comparing results}
\framesubtitle{Clairvoyant learner}

\todo{Our implementation for the clairvoyant learner - also mention non determinism and blueprint system}

\end{frame}

% ----------------------------------------

\begin{frame}

\frametitle{Comparing results}
\framesubtitle{Regret}

\todo{General regret formulation: see slides 1-05 and 1-06}

\end{frame}

% ----------------------------------------

\begin{frame}

\frametitle{Gaussian processes}

\todo{GP description: see slides 3-05}

\end{frame}

% ----------------------------------------

\begin{frame}

\frametitle{UCB1 formulation}
\framesubtitle{Outline}

In the \textbf{UCB1} algorithm, every \textit{arm} is associated with an \textbf{upper confidence bound} which provides an \textit{estimation} of the reward gained by playing that specific arm.

When each arm has been played at least once in order to have a baseline for its reward, at every trial, the arm with the \textbf{highest upper confidence bound} is pulled (\textit{optimism in the face of uncertainty}).
After having collected the realization of the reward for the chosen arm, its \textit{upper confidence bound} is updated accordingly.

The main advantage of combining \textbf{gaussian processes} with \textbf{UCB1} is that, in this way it's possible to take advantage of the \textbf{GP}'s \textit{confidence interval} by modeling it as a \textbf{confidence bound}.

\end{frame}

% ----------------------------------------

\begin{frame}[fragile]

\frametitle{UCB1 formulation}
\framesubtitle{Formalities}

Code snippet that calculates the \textbf{upper bound}:

\begin{lstlisting}[style=Python, basicstyle=\tiny, numbers=none, framexrightmargin=-20pt]
def estimation(self):
  upper_bounds = (self.means + self.confidence * 1.96 * self.sigmas)
                   * self.normalize_factor
  return upper_bounds
\end{lstlisting}

The \textit{theoretical regret} given by this kind of algorithm is bounded by:

\begin{displaymath}
R(UCB1) \le \sum_{a:\mu_a < \mu_a^*} \frac{4log(T)}{\Delta_a} + 8\Delta_a
\end{displaymath}

where $\Delta_a$ is the difference in expected reward between the \\ optimal arm $a^*$ and the arm $a$: $\Delta_a = \mu_a^* - \mu_a$

\end{frame}

% ----------------------------------------

\begin{frame}

\frametitle{TS formulation}
\framesubtitle{Outline}

Some aspects of \textbf{TS} are similar to the \textbf{UCB1} since both are bandit algorithms, the main difference is that in \textbf{TS} every arm is associated to a $prior \beta distribution$.

Every arm has a prior on its \textbf{expected value} based on its mean distribution.
After drawing a sample according to the corresponding prior distribution, the algorithm chooses the arm with the \textbf{best sample}, then, it updates the distribution of the chosen arm according the observed realization.

Both \textbf{Thompson Sampling} (\textbf{TS}) and \textbf{Upper Confidence Bound} (\textbf{UCB}, in particular \textbf{UCB1}), in conjunction with \textbf{Gaussian Processes} are the base algorithms chosen for the learners in our project.

\end{frame}

% ----------------------------------------

\begin{frame}[fragile]

\frametitle{TS formulation}
\framesubtitle{Formalities}

The \textit{theoretical regret} given by this kind of algorithm is bounded by:

\begin{displaymath}
R(TS) \le (1+\epsilon) \sum_{a:\mu_a < \mu_a^*} \frac{\Delta_a(log(T)+log(log(T)))}{K*\Lambda(\mu_a*, \mu_a)} + C (\epsilon, \mu_{a_1} , ... , \mu_{a_{\left|A\right|}})
\end{displaymath}

where $K*\Lambda(mu_a*,mu_a)$ is the \textbf{Kullback-Leibler divergence} and, as before, $\Delta_a$ is defiened as the difference in expected reward between the optimal arm $a^*$ and the arm $a$: $\Delta_a = \mu_a^* - \mu_a$

\end{frame}

%----------------------------------------

\subsection{Contextual hypotesis}

% ----------------------------------------

\begin{frame}

\frametitle{Scenario}

We now assume that the binary features of the users cannot be observed and therefore data is considered as \textbf{aggregated}.

Since the features of the users are \textbf{not observable}, the $\alpha$ functions' shape for each class is unknown.

As a result, in our scenario the learner receives all the interactions minus the parameters of the $\alpha$ functions.

\end{frame}

% ----------------------------------------

\subsection{Algorithm}

% ----------------------------------------

\begin{frame}

\frametitle{Solving the problem}

\todo{how do we solve the problem}

\end{frame}

% ----------------------------------------

\begin{frame}

\frametitle{Algorithm outline}

Our \textbf{Alphaless Learner} creates 5 different \textbf{GP-MABs} (one for each product) with $n_budget_steps$ number of arms.
They learn and predict on the aggregated budget matrix and try to find the optimal allocation of the budget.

\todo{how the algorithm works}

\end{frame}

% ----------------------------------------

\subsection{Results}

% ----------------------------------------

\begin{frame}

\frametitle{}
\framesubtitle{}

\todo{results}

\end{frame}

% ------------------------------------------

\AtBeginSection[]
{
\begin{frame}{}
    \tableofcontents[sections={\thesection}]
\end{frame}
}

% Uncertain alpha functions and number of items sold ------------------------------------

\section{Uncertain $\alpha$-functions and number of items sold}

%-----------------------------------------

\subsection{Contextual hypotesis}

%-----------------------------------------

\begin{frame}

\frametitle{Scenario}

In this case the e-commerce website doesn't register neither the \textbf{units sold} for each product nor the \textbf{class parameters}.

\todo{expand}

\end{frame}

%---------------------------------------------

\subsection{Algorithm}

%---------------------------------------------

\begin{frame}

\frametitle{Algorithm}
\framesubtitle{Algorithm outline}

so in order to work it estimate the list of values, corresponding to the earnings for each product given the knowledge obtained by the learner until now
We separately consider all the aggregated interactions where users landed on a certain product, then compute the reward they generated.
This way we obtain reward associated to a single campaign.

\todo{expand and correct}

\end{frame}

% ----------------------------------------

\subsection{Results}

%-----------------------------------------

\begin{frame}

\frametitle{Results}
\framesubtitle{}

\todo{results}

\end{frame}

% ----------------------------------------

% Uncertain graph weights ---------------------------------------------------------------

%\section{Uncertain Graph Weights}

% Non-stationary demand curve -----------------------------------------------------------

%\section{Non-stationary demand curve}

% Context generation --------------------------------------------------------------------

%\section{Context generation}

\end{document}


%NOTE TO US:
%%\centering if we want the presentation to be power-point style
