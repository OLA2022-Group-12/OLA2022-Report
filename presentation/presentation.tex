%----------------------------------------------------------------------------------------
%	PACKETS AND CONFIGURATION
%----------------------------------------------------------------------------------------

\documentclass{beamer}

\usepackage{title}  % Title settings for the presentation

%----------------------------------------------------------------------------------------
%	DOCUMENT
%----------------------------------------------------------------------------------------

\begin{document}

% TITLE
\frame{\titlepage}

% Introduction

\begin{frame}
    \frametitle{Table of Contents}
    \begin{list}{*}{\setlength{\itemsep}{1cm}}
        \item<1-> Enviroment 
        \item<2-> Learners
        \item<3-> Algorithms
        \item<4-> Results
    \end{list}
    
\end{frame}


\begin {frame}
\frametitle{Enviroment }
In our framework the enviroment is generating the complete class of data which are fed to the learners after being properly masked
So the enviroment gives us a constant base on which we build the entire project. This helps us with:
 
\begin{list}{-}{\setlength{\itemsep}{0.5cm}}
        \item<1-> Reduce randomness
        \item<2-> Comparison of alghorithms    
    \end{list}
    since the input is the same and affected by the same randomness.
\end {frame}

\begin {frame}
\frametitle{Enviroment }
In the enviroment it has been modeled both the different classes of users and their interaction with the "e-commerce website".
This allows to collect disaggregated or aggregated rewards
\end{frame}

\begin {frame}
\frametitle{Learners}
To have a better understanding of the performances of our learners we also coded a "stupid" learner which every day randomly assigns budget and a clayrvoiant one.
The comparison with those gives a better understanding of the different learners as the data we feed them change.
\begin{exampleblock}{Obviously to the clairvoyant one is given the entire data to chose the best possible partition of budgets each time.}
    \end{exampleblock}
    
\end{frame}

\end{document}
