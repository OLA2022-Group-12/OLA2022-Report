\chapter{Introduction}
\label{chap:introduction}

This project was made for the \texttt{Online Learning Applications} class at \textbf{Politecnico di Milano} with the supervision of prof. Nicola Gatti during a.a. 2021-2022; the project features a common part to all the project proposals, which is related to \textit{social influencing}, and a proposal-specific argument that, in our case, is \textit{advertising}. 

\section{Overview}
\label{sec:overview}

The project should model an e-commerce website capable of offering to its customers 5 different possible products labeled as $P_i$ having $i \in (1,\,2,\,3,\,4,\,5)$.

Products are displayed inside web pages; a web page may have a \textit{primary product} (which can be bought directly) and two \textit{secondary products} organized as recommendations where the first \textit{secondary product} has higher priority than the second one, moreover, the price for \textit{secondary products} isn't shown but they have the capability of bringing the customer on their respective primary web page upon being clicked.

Every day, customers can land on the primary pages of the various products or on the web page of a competitor. 

The website is in need of advertising its own products and has a predetermined \textit{budget} to spend in order to boost the probability of possible customers landing on a page owned by the website. 

\section{Hypoteses}
\label{sec_hypoteses}

This section holds all of the general hypoteses that were given or were independently formulated in order to create a feasible and focused project.

This does not include specific hypoteses and assumptions that are strictly related to the computational or numerical side of the project, as they will be discussed in their respective chapters.

\begin{itemize}
    \item The customer effectively buys products only at the end of their visit to the website.
    \item A user buys an arbitrary amount of products of a certain type if the price of a single unit of product is strictly under their \textit{reservation price}.
    \item The website offers an infinite number of units for each product and a can customer buy an arbitrary number of them.
    \item A single user won't open two times the same web page containing the same product during a visit to the website.
    \item All of the actions performed by the users are perfectly observable by the e-commerce website.
    \item Every \textit{primary product} has a \textbf{fixed} pair of \textit{secondary products} that will be displayed in the modalities discussed above.
    \item The price of every product is \textbf{fixed} and equal to the margin.
    \item There is an \textit{advertising campaign} for each product (5 in total), and clicking on an ad brings the user to the web page containing the advertised \textit{primary product}.
    \item \textit{Bidding optimization} is outside the scope of the project.
\end{itemize}

\todo{add our additional hypoteses}