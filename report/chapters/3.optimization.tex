\chapter{Optimization algorithm}
\label{chap:opt_alg}

\section{Problem Formulation}
\label{sec:Opt_Problem Formulation}
Our problem, since the bidding part is out of the focus of the project, can be reduced to finding the optimal budget for each subcampaign in order to maximize the profit.

The profit is defined as the difference between the expected margin and the budget spent in advertising: in practice we can calculate it by multiplying the quantity that has been bought for the price of the respective product since in the project's assumption the price is equal to the margin.

The number of clicks is represented by $\alpha_i$ which are influenced by how much of the budget we invest on that particular subcampaign.
Basically, it is a maximisation problem subject to an obvious constraint: the sum of the daily budget can't be greater then the overall budget.

The optimization problem can be expressed as follows:

\begin{displaymath}
F=\max_{\substack{x_i \in B}} \sum_{i=0}^n \alpha_i(x_i)p_i-x_i \ s.t. \sum_{i=0}^n x_i\leq B
\end{displaymath}

Where:

\begin{itemize}
    \item F represents the profit that needs to be maximized with respect to the daily budget
    \item $\alpha_i$ are the value per click, set equal to the expected margin from landing to the corresponding product.
    \item B is the total budget that can be used.
    \item $x_i$ are the amount of money spent on the sub campaign ads for each product $P_i$.
\end{itemize}

We can use a dynamic programming approach to optimize this function considering that all the parameters are known, in this way we can find the best way to spend the budget, for every feasible value of the budget.

Furthermore we can observe that it's possible to neglect the direct dependence of the margin from the budget cost since it wouldn't affect the result of our optimization approach.

\section{Code Analysis}
\label{sec:Opt_Code Analysis}

We solve the budget assignment problem over the value matrix representing our optimization problem by using the \textit{budget\_assignment} function. The matrix is assumed to be divided into $N$ rows and $M$ columns. In the rows we have the sub campaigns, in the columns we have the values of the daily budget described by fractions from 0 to $B$, where $B$ is the max budget. The function considers that the sum of the returned allocations cannot exceed the previous column. The value in each cell is the reward of assigning budget j to campaign i and return a vector where row i has the index of selected column j of C.

\begin{lstlisting}[style=Python]
def budget_assignment(c):
    """Solve the budget assignment problem over the value matrix c.

    The matrix c is assumed to be divided into n rows (campaigns) and m columns
    (budget fractions). As the columns describe budget fractions from 0 to B,
    where b is the max budget, the function only needs to consider that the sum
    of the returned allocations/selections cannot exceed m - 1. To put it
    another way, column j repreresents the value of allocating j * (b / m) of
    the budget to the campaigns.

    Arguments:
        c: numpy matrix (nxm) where the value is the reward of assigning budget
           j to campaign i

    Returns:
        a numpy vector where row i has the index of selected column j of C
    """

    n, m = np.shape(c)
    if m == 0 or n == 0:
        return np.array([])

    # Setup a dp table (which will contain the computed cumulative value of a
    # certain allocation) and a table to store the allocations themselves. We
    # initialize both with the first input, as when there is nothing before us
    # we will always allocate the most possible to the single campaign.
    dp = [c[0]]
    allocs = [np.arange(m)]

    for i in range(1, n):
        next_dp_row = []
        next_alloc_row = []
        for j in range(m):
            # We reverse the previous row of the dynamic programming storage, to
            # easily compare the options we have by summing across the vectors
            prev_row_r = dp[i - 1][j::-1]
            curr_row = c[i][: j + 1]

            row_sum = prev_row_r + curr_row

            # The best index is the index which we should select for the maximum
            # profit, it will be what we store in allocation, and we will also
            # store the value of the row_sum at best_index to dp[i][j]
            best_index = np.argmax(row_sum)

            next_alloc_row.append(best_index)
            next_dp_row.append(row_sum[best_index])

        allocs.append(np.array(next_alloc_row))
        dp.append(np.array(next_dp_row))

    dp = np.array(dp)
    allocs = np.array(allocs)

    # We get the best allocation from the last row of the dp table
    best_dp_index = np.argmax(dp[-1])

    # As we stored the allocation of what the dp would mean, we now need to
    # trace back through the allocs array to create the final allocation
    last_alloc = allocs[-1][best_dp_index]

    # A list of the final allocations that will give the optimal budget
    # utilization
    final_allocs = [last_alloc]
    remaining_budget = m - last_alloc

    # This will loop backwards through allocs, excluding the last row, so from n - 2 to 0
    for i in range(n - 2, -1, -1):
        # Based on the remaining_budget, we take the max of the remaining
        # possible cumulative values that we have in the dynamic programming
        # table
        sub_dp = dp[i, :remaining_budget]
        sub_best_index = np.argmax(sub_dp) if len(sub_dp) > 0 else 0

        # Convert the index we chose into the actual allocation that was done
        # for the given sub_dp index
        next_alloc = allocs[i][sub_best_index]
        final_allocs.append(next_alloc)

        # As we might have taken more off the budget, we need to adjust it
        remaining_budget = remaining_budget - next_alloc

    # As we constructed the the final allocation from the back to front, we need
    # to reverse it before returning it
    return np.array(list(reversed(final_allocs)))
\end{lstlisting}

\section{Solving the Budget Assignment Algorithm}
\label{sec:budget_assignment_algorithm}

As a part of the optimization problem we utilize a dynamic programming algorithm to solve a discretized budget assignment problem. Assigning the budget in the most profitable way is similar to solving the multidimensional knapsack problem.

By first creating a matrix over the value of assigning a certain budget to a given sub-campaign. The rows of the matrix correspond to the different sub-campaigns and the columns corresponding to the different budget levels. For the algorithm to work, it is important that the columns are separated by a constant step size. As a simplification we have in our case then determined that column $j$ signifies a budget of $j * (B / M)$, where $B$ is the total budget and $M$ is the amount of columns. Put simply, the columns represent a budget fraction of the total budget.

Given this matrix, the algorithm will then find the best allocation of the budget to maximize the total value. By iterating over the rows of the matrix, and in essence taking more and more sub-campaigns into consideration, two new tables are created which store both the highest value so far and the needed allocation to achieve that value. E.g. when calculating the value for budget fraction $\frac{1}{2}$ when only the first sub-campaign is considered, the full amount of the current fraction would always be assigned to current, and only, campaign. However when the second campaign is introduced, all possible combinations of assignments between the two are considered. In that case there might be that allocating $0$ to the first and $\frac{1}{2}$ to the second gives the highest value. Then the value of this allocation is stored in the dynamic table at the second row and column of the budget fraction $\frac{1}{2}$. Lastly the second table is populated with the index of the allocation made, so in this case the index of the chosen budget fraction.

After constructing the two dynamic tables above, the last rows then consist of the situation in which all sub-campaigns are considered, hence by choosing the biggest value in this row, the best budget assignment for the entire problem is found. Using the table with the allocation indices the amount of the budget for the given campaign is stored, and the remaining budget is updated. I.e. if allocated half of the budget for the last campaign, we want to best distribute the remaining half of the budget to the other sub-campaigns.
