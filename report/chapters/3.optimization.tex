\chapter{Optimization algorithm}
\label{chap:opt_alg}

\section{Problem Formulation}
\label{sec:Opt_Problem Formulation}
Our problem, since the bidding part is out of the focus of the project, is reduced to finding the optimal budget per campaign in order to maximize the profit.
The profit is defined as the difference between the (expected margin) and the spent in advertising and it is given by the number of each product bought and (his price); since in the project's assumption the price is equal to the margin.
The number of clicks is represented by $\alpha_i$ which are influenced by how much of the budget we invest on that particular subcampaign.
Basically, it is a maximisation subject to the obvious constraint: the sum of the daily budget can't be greater then the overall budget.

\todo{clarify why price and not expected return of product,Moreover we decided to not put any lower and upper bound on subcampaign to better explore all possibilities?}

The optimization problem can be expressed as follows:
\begin{displaymath}
F=\max_{\substack{x_i\in B}} \sum_{i=0}^n \alpha_i(x_i)p_i-x_i \ s.t. \sum_{i=0}^n x_i\leq B  
\end{displaymath}
\todo{how to nicely space formulas?}

Where: 
\begin{itemize}
    \item F represents the profit defined as the difference between the expected margin and the spent in advertising.
    \item $\alpha_i$ are the value per click set equal to the expected margin from landing to the corresponding product.
    \item B is the total budget usable.
    \item $x_i$ are the money spent on the sub campaign ads for each product $P_i$.
\end{itemize}

At start we can use a dynamic-programming algorithm to optimize this function considering that all the parameters are known.
\todo{and neglecting that the budget spent is to subtract from the objective function} Furthermore, in the dynamic-programming algorithm, we can find the best way to spend the budget, for every feasible value of the budget. 
\todo{This can give us a clairvoyant result with witch we can compare other more realistic algorithm.?}

\section{Code Analysis}
\label{sec:Opt_Code Analysis}

By the aid of the function \textit{budget assignment} we solve the budget assignment problem over the value matrix representing our optimization problem. The matrix is assumed to be divided into N rows and M columns. In the rows we have the sub campaigns, in the columns we have the values of the daily budget described by fractions from 0 to B, where B is the max budget. The function considers that the sum of the returned allocations can not exceed the previous column. The value in each cell is the reward of assigning budget j to campaign i and return a vector where row i has the index of selected column j of C.

\section{Solving the Budget Assignment Algorithm}
\label{sec:budget_assignment_algorithm}

As a part of the optimization problem we utilize a dynamic programming algorithm to solve a discretized budget assignment problem. Assigning the budget in the most profitable way is similar to solving the multidimensional knapsack problem.

By first creating a matrix over the value of assigning a certain budget to a given sub-campaign. The rows of the matrix correspond to the different sub-campaigns and the columns corresponding to the different budget levels. For the algorithm to work, it is important that the columns are separated by a constant step size. As a simplification we have in our case then determined that column $j$ signifies a budget of $j * (B / M)$, where $B$ is the total budget and $M$ is the amount of columns. Put simply, the columns represent a budget fraction of the total budget.

Given this matrix, the algorithm will then find the best allocation of the budget to maximize the total value. By iterating over the rows of the matrix, and in essence taking more and more sub-campaigns into consideration, two new tables are created which store both the highest value so far and the needed allocation to achieve that value. E.g. when calculating the value for budget fraction $\frac{1}{2}$ when only the first sub-campaign is considered, the full amount of the current fraction would always be assigned to current, and only, campaign. However when the second campaign is introduced, all possible combinations of assignments between the two are considered. In that case there might be that allocating $0$ to the first and $\frac{1}{2}$ to the second gives the highest value. Then the value of this allocation is stored in the dynamic table at the second row and column of the budget fraction $\frac{1}{2}$. Lastly the second table is populated with the index of the allocation made, so in this case the index of the chosen budget fraction.

After constructing the two dynamic tables above, the last rows then consist of the situation in which all sub-campaigns are considered, hence by choosing the biggest value in this row, the best budget assignment for the entire problem is found. Using the table with the allocation indices the amount of the budget for the given campaign is stored, and the remaining budget is updated. I.e. if allocated half of the budget for the last campaign, we want to best distribute the remaining half of the budget to the other sub-campaigns.

\todo{Should we include the actual code too as a listing here?}
\todo{Should there be a visual explanation aswell (it's hard to explain in text tbh)?}
