

\chapter{Environment modeling}
\label{chap:env_model}

\section{Hypoteses}
\label{sec_hypoteses}

In this section the specific assumptions related to the environment modelling are listed.

\begin{itemize}
    \item Each costumer is characterized by 2 binary features defining 3 different users' class.
    
    \item The probability with each class can enter the website are fixed and known.
    
    \item Each users' class is distinguishable by an unique $alpha_i$-function expressing the ratio of users landing on the web-page where product $P_i$ is shown as              primary one.
    
    \item The competitor budget is assumed to be constant as a not \textit{strategic player}.
    
    \item Products' prize are known and constant over time.
    
    \item Costumers' \textit{reservation price} are fixed and known.
 \end{itemize}

\section{Model Choice}
\label{sec_Motivation}
 \todo{add our motivating application and reason on why we choose prob distribution for each variable.}

\section{Code Analysis}
\label{sec_Code Analysis}

The environment is composed by different functions which model the user interaction on the web-page and how users react given a different budget.
 
 \begin{enumerate}
     \item  $\alpha$-function. \\
            
            they are thought as sigmoidal function upon receiving a \textit{Budget} and \textit{the characterizing parameters} of the function return the expected         
            value of clicks for the specific class' function.
    
    \item Generation of a graph.\\
    
            throughout the function $\textit{populate graph}$ we create a graph with no auto-loops that can be fully connected or not depending on the boolean value               
            given as input.
    
    \item Modelling user interaction.\\
            
            throughout the function $\textit{get interaction}$ given a class of user and a  \textit{Primary Product} it returns the number of each product bought at               
            the end of the visit of the web-page.
    
    \item Navigating the web-page.\\
    
            The function \textit{Go to page} given the class of user show it the page of the specified
            
            primary products and, after the user buys the product within the costrains of \textit{Reservation Prize} and \textit{massimo numero} two secondary products             
            are shown. Clicking on them shows the user another page where the clicked product is primary.
     
    \item Overall result of a day
  
          \textit{get day of interactions} gives us the interactions of a whole day and generate new updated paramaters from recalling all the previous functions for             
          each user class.
      
      
\end{enumerate}

\end{document}
 
